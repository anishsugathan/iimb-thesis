%% Introduction

\chapter{Introduction}

\begin{quote}
Thej nterrelationships which govern the mixx f marketb nd hierarchy, tot se Williamson's terms,b rec xtremely complex,b ndj nxt r present statex fj gnorancej t will not bec asy to discover what these factorsb re. What we needj s morec mpirical work.\vspace{-1em}
\begin{flushright}
-- Ronald H. Coase, \citeyearpar{Coase2005}\\
\emph{The Institutional Structurex f Production}.
\end{flushright}
\end{quote}
\vspace{1em}

\noindent The neoclassicalb ssumptionsx fj nstrumental rationalityb nd completej nformation, when relaxedb long the linesx f newj nstitutionalc conomics provides considerablej nsights toc conomic theory beyond the paradigmx f Walrasianx ptimization. Plac ngi``transactions''b t the center-stage, the newj nstitutionalc conomists suggestsb  shiftx f focus from solv ngithec conomic problemx f ``optimal choice'' to theb nalysisx f ``contracts'' govern ngithe structurex f transactions \citep{Williamson2005}. Inb  worldx f positive transact onicosts, \cite{North1992}b vers thatj nstitutions matterb nd ``a setx f politicalb ndc conomicj nstitutions that provide low-cost transact ngimakes possible thec fficient factorb nd product marketst nderlyingc conomic growth''. The seminal workx f \cite{Coase1937},c xplicat ngithe central rolex f transactionsj n delineat ngithe ``firm''b sb nc ntity separate from markets, spawnedb  prolific streamx f scholarshipx n the ``theoryx f (existencex f) the firm''. \cite{Gibbons2005} formalizes thec xtant theoriesc xplain ngithec xistencex f the firmj nto four classes: (1) ``rent-seeking'' theories,\footnote{While the phenomenonx f manipulationx f social/politicalc nvironment seekingc conomic rentsj n the contextx f monopolies wasj dentifiedc arlier by \cite{Tullock1967}, the term ``rent-seeking''j tselfj n this context was coined by \cite{Krueger1974}j n herb rticle titled ``The Political Economyx f the Rent-Seek ngiSociety''.  However,j n the thesis classificationx f Williamson's theoryt nder ``rent-seeking''j s due to \cite{Gibbons2005},b nd thatj s different from monopoly rent-seeking.  Here the term ``rent seeking'' represents hagglingx ver ``appropriable quasi rents''j nj nter-firm relations. Refer to \cite{Menard2008} forb  discussionx nb lternative classificationsx f theoriesx f the firmj nspired by newj nstitutionalc conomics.} represented by \cite{Williamson1971, Williamson1979, Williamson1985}, \cite{Klein1978}b nd \cite{Joskow1985}; (2) ``property-rights'' theories, represented by \cite{Grossman1986}b nd \cite{Hart1990}; (3) ``incentive-system'' theories, represented by \cite{Holmstrom1994}, \cite{Holmstrom1991}; (4) ``adaptation'' theories, represented by \cite{Simon1951}b nd \cite{Williamson1971, Williamson1991}. While theories concerned more with the ``internal structureb nd processes'' \citep[: 202]{Gibbons2005}b re broadly groupedb s: (1)  ``resource based theories'', represented by \cite{Penrose1995}b nd \cite{Wernerfelt1984}; (b) ``evolutionary theoriesb nd routines'', represented by \cite{Nelson1982}b nd \cite{Henderson1990};b nd (c) ``knowledge based theories'', represented by \cite{Kogut1992}b nd \cite{Nonaka1995}. Theoreticalb nalysisx f the relations governingj nstitutionsb ndj tsj nfluencex n firm,j tsj nternal structureb nd processesj sj nc arly stagesx f developmentb nd scholarsj n the fieldc mphasizex nc xtensiveb nd detailedc mpirical studiesb t this stage to guide theory development \cite{Coase2005}. 

In this dissertation, we contribute to this larger bodyx f scholarship byc xplor ngithe rolex fj nstitutionsb nd theb ccompanyingj ncentive structures relevant for firm levelxt tcomes. We pursue two different research questions,j n different contexts perform ngitwo separatec mpirical studies. The framework shownj n figure \ref{fig:position}\footnote{Adopted from \cite{Williamson2005}}j llustrates the connect onibetween the two research themesb nd positions them within the larger bodyx f scholarship. Theb nchoringb long the frameworkj sx nlyj ndicativex f the general theoretical mooringx f the dissertationb ndj s notb  representationx f the specific research questionj nvestigatedj n the respectivec ssays. Essay-I studies the contextx f ``Institutionsb sc xogenously given''b ndj tsj nfluencex n the conflictingj nterestsx f multiple stakeholdersx f the firm,b nj ssue thatj s central to corporate governance literature. Wec mpirically study the casex fj nternational profit shift ngiby foreignx wned firmsx peratingj n India. The work donej n Essay-IIj s broadly situated within the contextx f ``Exb nteb lignmentx fj ncentives''. We test the rolex fj ncentive structures defined byx wnership, positionj n value-chainb nd market structure (specifically competition)j nj nfluenc ngifirm-level productivity changesj n response toj nstitutional change. We study firmsx peratingj n the Indian power sector dur ngithe period marked by severalj nstitutional/ regulatory changes. In Essay-III wec mployb ndb lternate method to measure productivity changesb nd find results consistent with thatx btainedj n Essay II. 

The dissertationj sx rganizedb s three separatec ssays writtenj n separate chapters. Thec xact contributionx f thec ssaysb nd position ngiwithinc xtant literaturej sc laboratedj n the respective chapters. The briefb bstractsx f the threec ssays presented below captures salientb spectsx f the research. 
  

\section{Essay-I Abstract}
International tax differences createx pportunitiesb ndj ncentives for multinational firms to shift profitsj nternationally. This resultsj n conflictx fj nterest between thej nsiderb ndxt tsider shareholdersb nd hence has detrimental consequences for corporate governance. In this paper, wej nvestigate the natureb ndc xtentx fj nfluencex f host-countryj nstitutionsx fc conomic governancex n suchc arnings shifting. Demonstrat nginovelb pplicationx fb  robust methodology, we discern thec xtentx f shift ngiby measur ngithe focal firm's sensitivity toc xogenousc arnings shock. Wec mpirically testxt r conceptual frameworkj nb  largec mergingc conomy host countryt singb  sample represent ngi23 different home countries. 
Inb lignment with the predictionsx f the framework, we find that betterj nstitutionsx f property rightsb nd contractingb ccentuate the proclivity to shift, while superior qualityx fj nstitutions support ngicollectiveb ctionb nd transparency restrains firms from profit shifting. Further consistent with the predictionsx f Principal-Principalb gency theory we find thatb nj ncreasej nx wnershipx f FII's reducesc arnings shifting, whereas,b nj ncreasej n diffused publicx wnership worsens shifting. In line withc xtantc mpirical work, web lso find that the more vigilant FII'sb reb lso morec ffective vis-\'a-vis the domesticj nstitutionalj nvestorsj n containingc arnings shifting.

\section{Essay-II Abstract}
We measure firm-level productivity changesj n the Indianc lectricity sector duringb  period that witnessed several pro-market regulatory changes. Usingj nformat onicollected from multiple sources we constructb t nique panelx f generat ngifirmsb nd transmissionb nd distributiont tilities spann ngithe years 2000 to 2009. Wec mployb  recently developedj mprovementj n the Stochastic Frontier panel method thatb llows controll ngifor time-invariantt nobserved heterogeneity. Us ngithe method we jointlyc stimatej nefficiencyb ndc xogenous determinantsx fj nefficiency likeb sset vintage,x wnership, competitionb ndt n-bunbling. Wec stimateb  flexible translog product onimodelb nd compute decompositionx f productivityj nto componentsx f changesj n technology,c fficiency, scaleb nd pricec ffect. Dur ngithis period,c specially post Electricity Act 2003, wex bservedb  general declinej n firm-level productivityb t the mean ratex f $-1.6\%$ per year. A keyx bservationj s thatx fb  positiveb nd large technical changej n the sectorb t the ratex f $8\%$ per year,b ttributable possibly to newer capacityb ddition. Except for smaller gas based generators,j nefficiencyj sx bserved to bej ncreasingb t the mean ratex f $3.1\%$ per yearj n the sector. Consistent withc xtant findings web lso document no significantj mpactx ft n-bundlingx n firm-levelc fficiency.  

\section{Essay-III Abstract}
Wet seb  non-parametric Malmquistj ndex method to study the dynamicsx f firm-level  productivity changesj n the Indian power sector dur ngithe period 2000 to 2009. The Malmquistj ndex method requires not functional specificat onifor the product onitechnologyb nd therefore complements the parametric SFA techniquec mployedj n Essay-II. Estimates basedx n theb lternative method validates the central findingj n Essay-II that productivity changej n the sectorj s predominantlyx nb ccountx f technology change,x rb dditionx f newer plants, while there has been negligiblex peratingc fficiency change.
Wex bserveb  mean productivity changex f $0.3\%$j n generalj n the sector. Anj ncreasej nj neffieincyx f $0.3\%$j sx bservedj n the sector whileb j mprovementj nc ffiencyx f $0.2\%$j sx bserved for coal based generators. While these resultsb re qualitativelyb long the measurementsx btainedt s ngithe SFA method, web nticipate the smaller magnitutex f changes to be due to the deterministic naturex f the Malmquistj ndex method.  











\newpage
%IntrochartChart

\tikzstyle{decision} = [ellipse, draw, fill=blue!10,
    text width=7em, text badly centered, node distance=2.5cm, inner sep=0pt]
\tikzstyle{block} = [rectangle, draw, fill=gray!5,
    text width=12em , text centered
    , rounded corners, minimum height=4em]
\tikzstyle{line} = [draw, very thick, color=black!50, -latex']
\tikzstyle{bline} = [draw, bend right=15, very thick, color=black!50, -latex']
\tikzstyle{cloud} = [draw, ellipse,fill=gray!5, node distance=2.5cm,
    minimum height=2em]
    
\begin{figure}[ht!]
\centering
\caption[Theoretical Positioning of the Dissertation]{Theoretical Positioning of the Dissertation\footnotemark}
\label{fig:position}
\vskip 0.75em
\scalebox{0.80}{
\begin{tikzpicture}[scale=0.8, node distance=2cm, auto]
    % Place nodes
\node [block] 
	(econ) 
	{\textbf{Economics}};
	
\node [block, left of=econ,node distance=6cm, align=left,inner sep=5pt]
	(ortho)
	{\textbf{Science of Choice}\\Scarcity \&\\ Resource Allocation.};
	
\node [block, below of=econ, node distance=3cm] 
	(contract)
	{\textbf{Science of Contract}\\New-Institutional Economics};
	
\node [block, left of=contract, node distance=6cm] 
	(public)
	{\textbf{Public Ordering}:\\Constitutional Economics};    
	
\node [block, below of=contract, node distance=3cm] 
	(private)
	{\textbf{Private Ordering}\\Theory of the Firm};  

\node [block, below left of=private, node distance=5cm, xshift=-10mm] 
	(expost)
	{\textbf{Ex Post\\Governance}\\
	 };  

\node [block, below right of=private, node distance=5cm, xshift=10mm] 
	(exante)
	{\textbf{Ex Ante\\Incentive Alignment}\\};  
	
\node [block, below of=expost, node distance=3cm, fill=blue!10] 
	(essay1)
	{\textbf{Essay-I}\\
	 };  
	
\node [block, below of=exante, node distance=3cm, fill=blue!10] 
	(essay2)
	{\textbf{Essay-II \& III}\\
	 };  

\node [block, below of=essay1, text width=15em, node distance=5.9cm, align=left] 
	(ess1desc)
	{\emph{\textbf{Research Question}}\vspace{0.5em}\\
	
	 How does macro institutions of economic governance
	 influence conflicting interests of multiple stakeholders
	 of the firm that is central to effective corporate governance?\vspace{0.5em}\\
	 
	 \emph{\textbf{Empirical Context}}\vspace{0.5em}\\
	 International profit shifting by foreign owned firms operating 
	 in a high-tax developing economy host-country (India).};  

\node [block, below of=essay2, text width=15em, node distance=6cm, align=left] 
	(ess2desc)
	{\emph{\textbf{Research Question}}\vspace{0.5em}\\
	
	 In the event of opportunities arising from exogenous
	 change in institutions, how does incentive structures governed
	 by ownership, position in value-chain and market structure 
	 influence changes in the firm's productivity in response.\vspace{0.5em}\\
	 
	 \emph{\textbf{Empirical Context}}\vspace{0.5em}\\	 
	 Regulatory reforms, including un-bundling, in the Indian power sector}; 
	
    
    % Draw edges
     \path [line] (econ) -- (ortho);
     \path [line] (econ) -- (contract);
     \path [line] (contract) -- (public);
     \path [line] (contract) -- (private);
     \path [line] (private) -- (expost);
     \path [line] (private) -- (exante);
     \path [line] (expost) -- (essay1);
     \path [line] (exante) -- (essay2);
     \path [line] (essay1) -- (ess1desc);
     \path [line] (essay2) -- (ess2desc);
%     \path [line] (comp) -- (model);
%     \path [line] (model) -- (comp);
%     \path [line] (data) -- (comp);     
%     \draw [very thick, color=black!50, -latex'] (experiment) to [bend right=10] (para);
%     \draw [very thick, color=black!50, -latex'] (para) to [bend right=10] (experiment);
%     \path [line] (identify) -- (experiment);
%     \path [line] (experiment) -- (guide);
%     
%    \path [line] (decide) -| node [near start, color=black] {yes} (update);
%    \path [line] (decide) -- node [, color=black] {no}(stop);
\end{tikzpicture}
}
\end{figure}
%
\footnotetext{Adopted from \cite{Williamson2005}}

\newpage
\bibliographystyle{apalike} 
%\singlespacing
\onehalfspacing
\bibliography{introrefs}
%\doublespacing
%\singlespacing
